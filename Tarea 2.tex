\documentclass{article}
\usepackage[utf8]{inputenc}

\usepackage{hyperref}

\hypersetup{
    colorlinks=true,
    linkcolor=blue,
    filecolor=magenta,      
    urlcolor=cyan,
    pdftitle={Overleaf Example},
    pdfpagemode=FullScreen,
    }
    
\urlstyle{same}
    
\title{Principios de Seguridad en Sistemas Operativos \\ Tarea 2 Race Condition Vulnerability}
\author{Juan Jose Araya Castro \\
Alvaro Andrei Miranda Muñoz}
\date{August 2022}

\begin{document}

\maketitle

\begin{minipage}{\textwidth}
Código curso: MC1005\\
Due: Agosto 17, 2022 at 23:59 (CST)\\
Total puntos: 10 puntos\\
\end{minipage}
    
\section*{Introducción}
Condicion de carrera

\section*{Instrucciones Generales}

\begin{enumerate}
  \item 
  \item 
  \item 
  \item 
  \item 
  \item 
  
\end{enumerate}


\section*{Descripción del Ataque}

\tab El ataque de Vulnerabilidad de Condición de Carrera\\

\\
\\
Según Wikipedia ...:

\begin{enumerate}
  \item  
  \item 
\end{enumerate}

\begin{enumerate}

  \item 
  
\end{enumerate}

\section*{Documentación del Ataque}

Videos de las ejecuciones y los ataques: \\

\begin{enumerate}
  \item Actividad 1.1:

  \href{https://www.loom.com/share/622ec72dc2cb4c658d06e8dee2589afd}{Alvaro Miranda - Juan Jose Araya - Actividad 1.1} \\
  
  \href{https://www.loom.com/share/7ac21ee2b903425195a60f802dd6a9c9}{Alvaro Miranda - Juan Jose Araya - Actividad 1.2} \\
  
  \item Actividad 2:
  
  \href{https://www.loom.com/share/36b82066d6d043168116ed0fc0b293b6}{Alvaro Miranda - Juan Jose Araya - Actividad 2.1} \\
  
  \href{https://www.loom.com/share/6dc3ff3fbb97469085bb2e26dcbf3220}{Alvaro Miranda - Juan Jose Araya - Actividad 2.2} \\
  
  \href{https://www.loom.com/share/466ba0ef96464f9a95a7eb8b19e8fb2f}{Alvaro Miranda - Juan Jose Araya - Actividad 2.3} \\
  
  \item Actividad 3:
  
  \href{https://www.loom.com/share/ae3e916e6bf14555b860367652aa316d}{Alvaro Miranda - Juan Jose Araya - Actividad 3} \\
  
\end{enumerate}



\section*{Autoevaluación}

\begin{enumerate}
    \item Estado Final: Completado
    \item Problemas Encontrados: ...
    \item Limitaciones Adicionales: ...
    \item Reporte de Commits en Git: \href{https://github.com/varitomirandacr/SeguridadSO_RaceCondition/commits/main}{Commits History}
    \item Evaluación: 100%
\end{enumerate}

    \noindent\rule{10cm}{0.4pt} \\
    Rubro   \qquad \qquad \qquad \qquad Puntaje Total    \qquad  Puntaje Obtenido \\
    \noindent\rule{10cm}{0.4pt} \\
    Actividad 1: \qquad \qquad \qquad \qquad \ 25 \qquad \\ 
    \noindent\rule{10cm}{0.1pt} \\
    Actividad 2: \qquad \qquad \qquad \qquad \ 25 \qquad \\ 
    \noindent\rule{10cm}{0.1pt} \\
    Actividad 3: \qquad \qquad \qquad \qquad \ 25 \qquad \\ 
    \noindent\rule{10cm}{0.1pt} \\
    Documentación del Ataque: \quad \ \ 25 \qquad \\ 
    \noindent\rule{10cm}{0.1pt} \\

\section*{Lecciones Aprendidas}

Este proyecto de Vulnerabilidad de Condición de Carrera...:

\begin{enumerate}
    \item     
    \item     
    \item     
    \item 
    \item 
    
\end{enumerate}

\section*{Bibliografía}

\begin{itemize}
  \item Laboratory for Computer Security Education. \href{https://www.utc.edu/sites/default/files/2021-04/race-condition.pdf}{UTC EDU Race Condition}
  
  \item Lecture Notes (Syracuse University). \href{https://web.ecs.syr.edu/~wedu/Teaching/cis643/LectureNotes_New/Set_UID.pdf}{Set-UID Privileged Programs}
  
  \item SEED Labs – Race Condition Vulnerability Lab. \href{https://seedsecuritylabs.org/Labs_16.04/PDF/Race_Condition_new.pdf}{Race Condition Vulnerability Lab}
  
  \item Race Condition Vulnerability \href{https://moodle1920.up.pt/pluginfile.php/132925/mod_resource/content/1/Race_Condition_Extra.pdf}{Race Condition Vulnerability}
  
\end{itemize}

\end{document}
